\documentclass[12pt]{article}
\usepackage{amsmath, amssymb, graphicx, hyperref, float}
\usepackage{geometry}
\geometry{a4paper, margin=1in}
\title{Project 1: A Mathematical Investigation of Populations and Predator-Prey Dynamics}
\author{Group Members: Isaias Perez \\
Seamus Healy\\
Marius Grimsland \\ APPM 2360 - CU Boulder}
\date{\today}

\begin{document}

\maketitle

\begin{abstract}
This report investigates population dynamics through mathematical modeling, focusing on predator-prey interactions using the logistic equation and Lotka-Volterra systems. The study utilizes analytical methods and numerical simulations to explore population equilibria and stability.
\end{abstract}

\section{Introduction}
Provide a brief introduction to the project, highlighting the motivation behind studying predator-prey dynamics and the significance of understanding population changes over time. Mention key concepts, such as carrying capacity, intrinsic growth rate, and predation effects.

\section{Modeling Individual Populations: The Logistic Equation}
\subsection{Task Set A}
\begin{enumerate}
    \item Discuss the units of the parameters $r$ and $L$.
    \item Derive and present the equilibrium solutions of the logistic equation:
    \begin{equation}
        \frac{dx}{dt} = r\left(1 - \frac{x}{L}\right)x
    \end{equation}
    Include separation of variables and the explicit solution $x(t)$.
    \item Numerical solutions using Euler's method:
    \begin{itemize}
        \item Present plots comparing solutions for different step sizes $h$.
        \item Discuss accuracy and efficiency.
    \end{itemize}
    \item Classify the equation with harvesting:
    \begin{equation}
        \frac{dx}{dt} = r\left(1 - \frac{x}{L}\right)x - \frac{px^2}{q + x^2}
    \end{equation}
    \item Analyze the behavior of $H(x)$ graphically.
    \item Numerical analysis for different initial conditions.
\end{enumerate}

\section{Modeling Population Interactions: The Lotka-Volterra System}
\subsection{Task Set B}
\begin{enumerate}
    \item Classify the Lotka-Volterra system:
    \begin{align}
        \frac{dx_1}{dt} &= -\alpha x_1 + \beta x_1 x_2 \\
        \frac{dx_2}{dt} &= \gamma x_2 - \delta x_1 x_2
    \end{align}
    \item Derive nullclines and equilibrium points.
    \item Present vector field plots and analyze phase plane behavior.
    \item Discuss component curves $x_1(t)$ and $x_2(t)$.
\end{enumerate}

\section{The Logistic Predator-Prey Equations}
\subsection{Task Set C}
\begin{enumerate}
    \item Derive nullclines and equilibrium solutions analytically.
    \item Use numerical simulations to analyze system behavior for different initial conditions.
    \item Discuss stability and compare solution curves.
\end{enumerate}

\section{Model Comparison}
Compare and contrast the Lotka-Volterra and Logistic Predator-Prey models. Discuss their strengths, weaknesses, and possible modifications to improve model accuracy.

\section{Conclusion}
Summarize key findings, highlighting significant results and insights gained from the investigation. Reiterate the importance of mathematical modeling in understanding predator-prey dynamics.

\section*{Appendix}
\subsection*{Code}
Include all relevant code here, formatted properly and explained briefly where necessary.

\subsection*{Extended Calculations}
Present any lengthy derivations, additional graphs, or numerical solutions that were omitted from the main body of the report.

\section*{References}
\begin{enumerate}
    \item Predators and Prey: A Case of Imbalance between Mountain Lions and the North Kings Deer Herd. Johnston Ridge Observatory - US Forest Service. \\ \url{https://www.fs.usda.gov/psw/publications/Popular/mtnlions.html}
    \item Predator-Prey Dynamics: Lotka-Volterra. \\ \url{http://www.tiem.utk.edu/~gross/bioed/bealsmodules/predator-prey.html}
\end{enumerate}

\end{document}

